\documentclass[a4paper,10pt]{article}

%A Few Useful Packages
\usepackage{marvosym}
\usepackage{fontspec}                     %for loading fonts
\usepackage{xunicode,xltxtra,url,parskip}     %other packages for formatting
\RequirePackage{color,graphicx}
\usepackage[usenames,dvipsnames]{xcolor}
\usepackage[big]{layaureo}                 %better formatting of the A4 page
% an alternative to Layaureo can be ** \usepackage{fullpage} **
\usepackage{supertabular}                 %for Grades
\usepackage{titlesec}                    %custom \section

%Setup hyperref package, and colours for links
\usepackage{hyperref}
\definecolor{linkcolour}{rgb}{0,0.2,0.6}
\hypersetup{colorlinks,breaklinks,urlcolor=linkcolour, linkcolor=linkcolour}

%FONTS
\defaultfontfeatures{Mapping=tex-text}
%\setmainfont[SmallCapsFont = Fontin SmallCaps]{Fontin}
%%% modified for Karol Kozioł for ShareLaTeX use
\setmainfont[
    SmallCapsFont = Fontin-SmallCaps.otf,
    BoldFont = Fontin-Bold.otf,
    ItalicFont = Fontin-Italic.otf
]
{Fontin.otf}
%%%

%CV Sections inspired by:
%http://stefano.italians.nl/archives/26
\titleformat{\section}{\Large\scshape\raggedright}{}{0em}{}[\titlerule]
\titlespacing{\section}{0pt}{3pt}{3pt}

%Italian hyphenation for the word: ''corporations''
\hyphenation{im-pre-se}

%-------------WATERMARK TEST [**not part of a CV**]---------------
\usepackage[absolute]{textpos}

\setlength{\TPHorizModule}{30mm}
\setlength{\TPVertModule}{\TPHorizModule}
\textblockorigin{2mm}{0.65\paperheight}
\setlength{\parindent}{0pt}

%--------------------BEGIN DOCUMENT----------------------
\begin{document}


\pagestyle{empty} % non-numbered pages

% \font\fb=''[cmr10]'' %for use with \LaTeX command

%--------------------TITLE-------------
\par{\centering
    {\Huge Harsh \textsc{Gupta}
}\bigskip\par}

%--------------------SECTIONS-----------------------------------
%Section: Personal Data
\section{Personal Data}

\begin{tabular}{rl}
    \textsc{email:}     &

     \href{mailto:mail@hargup.in}{mail@hargup.in},
     \href{mailto:gupta.harsh96@gmail.com}{gupta.harsh96@gmail.com}\\
     % \href{mailto:harsh.gupta@iitkgp.ac.in}{harsh.gupta@iitkgp.ac.in} \\
    \textsc{LinkedIn}     &
    \href{https://in.linkedin.com/in/hargup}{https://in.linkedin.com/in/hargup}\\
    \textsc{GitHub}     & \href{https://github.com/hargup}{https://github.com/hargup}
\end{tabular}

%Section: Work Experience
\section{Work Experience}
\begin{tabular}{r|p{12cm}}

    % Emphasize on the work you did, Use bullets

    \emph{June 2015 - } & Wikipedia Human Gender Index (WHGI) \\\emph{March 2016}&
    \emph{\href{https://meta.wikimedia.org/wiki/Grants:IEG/WIGI:_Wikipedia_Gender_Index}{Individual
    Engagement Grant, Wikimedia}}\\& \\&
            Wikipedia Human Gender Indicators (WHGI)
            is a Wikimedia sponsored project which aims to
            develop statistical and quantitative indicators for gender gap
            and raise awareness by observing the trend of gender in biography
            articles.
        \begin{itemize}
            \item Wrote data pipeline and analysis to compare the occupational
                gender gap in statistics obtained from US Bureau of Labor and
                Statistics and the data obtained from wikidata.
            \item Created data pipelines and an interactive portal to display data analysis.
            \item Analyzed and compared the occupational data from Wikipedia
                biographies with the data provided by US Bureau of Labor Statistics
        \end{itemize}

\\\multicolumn{2}{c}{} \\


\textsc{Summer 2015} & Summer Intern at
\textsc{\href{http://continuum.io/}{Continuum Analytics}}
    \emph{}\\&
    \begin{itemize}
        \item Automated the process of building noarch conda packages from Python
    Packaging Index (PyPI). Conda is the package manager which powers the
    popular python distribution Anaconda.
        \item Added support for numpy.median in Numba, a NumPy aware optimizing compiler
    for Python.
    \end{itemize}

\\\multicolumn{2}{c}{} \\

% \textsc{Summer 2015} &
% \textbf{\href{https://github.com/sympy/sympy/wiki/GSoC-2015-Report-Amit-Kumar-:-Solvers}{Google
% Summer of Code Mentor}}, \\& \\&
%         Mentored a student in improving the symbolic
%         solvers in SymPy, the pure python Computer Algebra System\\\multicolumn{2}{c}{} \\



    \textsc{December 2014} & Open Source Contributor to \href{http://afra.sbcs.qmul.ac.uk/}{Afra}
        \emph{}\\& \\&
        Afra is an underdevelopment project by Queen Mary
    University of London to crowdsource genome annotation.
\begin{itemize}
        \item Wrote Afra's automated test suite in JavaScript's Jasmine
            framework.
        \item Wrote a ruby script to export annotation submissions by users.
\end{itemize}\\\multicolumn{2}{c}{} \\




\textsc{Summer 2014} &
\href{https://github.com/sympy/sympy/wiki/GSoC-2014-Application-Harsh-Gupta:-Solvers}{Google
Summer of Code Student, SymPy}
\emph{}\\& \\&
Wrote the new solveset module of SymPy. These new
solvers provides a clear input and output interface, removes inconsistency
between real and complex solvers and supports infinite solutions.\\\multicolumn{2}{c}{} \\

% \textsc{Summer 2013} & Web Development Intern at BreatheArts \emph{}\\&\footnotesize{
%
% BreatheArts was an online platform to sell and purchase art. I implemented a login
% system using OpenID and fixed the password recovery system on Breathe Art's
% ASP.NET online art galleries.
%
% }\\\multicolumn{2}{c}{} \\
\end{tabular}




\section{Education}

\begin{tabular}{rl}
    \textsc{2012-2017(expected)} & Master of Science, Bachelor of Science in
    \textsc{Mathematics} and \textsc{Computing}
    \\&7.53/10 \small\emph{CGPA},
    \normalsize\textbf{Indian Institute of Technology}, Kharagpur\\
\end{tabular}

\section{Publications}

\begin{itemize}
\item  \textbf{Monitoring the Gender Gap with Wikipedia Human Gender Indicators}
    (International Symposium for Open Collaboration 2016)\\
    \textit{Maximilian Klein, Harsh Gupta, Vivek Rai, Haiyi Zhu}
\end{itemize}


% \section{Other Projects}
%
% \begin{tabular}{rp{12cm}}
%     \textsc{} & \textbf{\href{https://wiki.metakgp.org}{MetaKGP}} \\&
%
%        A collaborative effort to create a wiki to collect and curate the vast
%        amount of campus specific information relevant to the student community
%        of IIT Kharagpur.
%
% \\\multicolumn{2}{c}{} \\
%
% %     \textsc{} & \textbf{\href{https://github.com/metakgp/mcmp}{mcmp}} \\&
% %
% %         A web interface where students can search professors of
% %         IIT Kharagpur by their research areas, this was done by scraping
% %         information from institute's website.
% %
% % \\\multicolumn{2}{c}{} \\
%
% \textsc{} & \textbf{Jigsaw Puzzle Solver} \\
% \textsc{} & \emph{Prof. S. K. Barai}, March 2014 \\&
%     Evaluated different techniques based on genetic algorithm to solve large
%     piece jigsaw puzzle (randomly shuffled pieces of an image); implemented
%     mutation strategies; came up with an approach to use this technique to solve
%     images with non unique components.
% \\\multicolumn{2}{c}{} \\
%
% \end{tabular}

\section{Positions of Responsibility}
\begin{tabular}{rp{12cm}}

    \textsc{} & \textbf{Mentor, Google Summer of Code 2016, SymPy}
    Co-mentored two students in successfully completing their Google Summer of
    Code projects with SymPy.
    \\&

\\\multicolumn{2}{c}{} \\


    \textsc{} & \textbf{Mentor, Google Summer of Code 2016, SymPy}
    Mentored a student in successfully completing their Google Summer of
    Code project with SymPy.
    \\&

\\\multicolumn{2}{c}{} \\

    \textsc{} & \textbf{Capitan, Data Analytics, Azad Hall of Residence}
    \\&
    Responsible for leading the hostel team in the inter hostel
    Data Analytics General Championship.
\\\multicolumn{2}{c}{} \\

\end{tabular}

\section{Talks}
\begin{itemize}
    \item \emph{Tech is WEIRD}, lightning talk at SciPy 2016, Austin, USA
    \item \emph{\href{https://www.youtube.com/watch?v=YCxQI4C34j8}{What's New with
        SymPy solvers}}, lighting talk at SciPy 2015, Austin, USA
    \item \emph{Symbolic Computation with Python using SymPy}, workshop at
        SciPy 2016, Austin, USA
    \item \emph{Symbolic Computation with Python using SymPy}, workshop at PyCon India
        2015, Bangalore, India
\end{itemize}

% \section{Scholarships}
% \begin{itemize}
%     \item Receive the annual scholarship INSPIRE from Department of Science and Technology.
%     \item Received a scholarship to attend the Annual SciPy conference at
%     Austin, TX
% \end{itemize}


% %Section: Languages
% \section{Languages}
% \begin{tabular}{rl}
%     \textsc{Italian:}&Mothertongue\\
%     \textsc{English:}&Fluent\\
%     \textsc{French:}&Basic Knowledge\\
% \end{tabular}

\section{Relevant Courses}
% \begin{tabular}{rl}
\begin{itemize}
\item Programming and Data Structure, Design and Analysis of Algorithms, Graph Theory and Algorithms
\item Computer Organization and Architecture, Systems Programming, Theory of Operating Systems
\item File Organization and Data Base Systems, Cryptography and Network Security
\item Probability and Statistics, Stochastic Processes, Computational Statistics
\item Numerical Solutions to ODE and PDE, Advanced Numerical Techniques, Linear Algebra, Operation Research
\item Discrete Mathematics, Modern Algebra, Functional Analysis, Switching and Finite Automata
\item Economics, Globalization and Culture, Constitutional Law - I, Introduction to Intellectual Property Law, Copyright, Laws of Patents - I
\end{itemize}


% \section{Skills}
% \begin{tabular}{rl}
%     Advanced Python, \LaTeX, Unix utilities,
%      Web Scraping, Data Visualization
% \end{tabular}

\end{document}
